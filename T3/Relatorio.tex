\documentclass[10pt,a4paper]{article}

\usepackage{indentfirst}
%\usepackage{arev}
\usepackage{amsthm,amsfonts,amsmath,amssymb}
\usepackage[brazilian]{babel}
\usepackage[T1]{fontenc}
%\usepackage[latin1]{inputenc}
\usepackage[utf8]{inputenc}
%\usepackage{multicol}
\usepackage{setspace}
%\usepackage{natbib}
\usepackage[usenames,dvipsnames]{xcolor} 
\usepackage{pgf,tikz}
%\usepackage{algpseudocode}
\usepackage{float}
\usepackage{graphicx}
\usepackage{subfigure}
\usepackage{wrapfig}
%\usepackage{listings}
%\usepackage[linesnumbered, ruled, portuguese]{algorithm2e}
\usepackage{multirow}
%\usepackage{verbatim}
%\usepackage[active,tightpage]{preview}
%\PreviewEnvironment{tikzpicture}
%\setlength\PreviewBorder{5pt}
\usepackage{geometry}
\usepackage[pdftex]{hyperref}
\usepackage{listings}
\usepackage[normalem]{ulem}
\geometry{a4paper,inner=2.0cm,outer=2.0cm,top=2.0cm,bottom=2.0cm}


\newcommand{\pr}{\hspace*{0.6cm}}
\newcommand{\vesp}{\vspace*{.3cm}}

\newcommand{\sen}{\mbox{\,sen}}
\newcommand{\cotg}{\mbox{\,cotg\,}}
\newcommand{\tg}{\mbox{\,tg\,}}
\newcommand{\cose}{\mbox{\,cos\,}}
\newcommand{\expo}{\mbox{\,e\,}}
\newcommand{\logg}{\mbox{\,log}}
\newcommand{\Sum}{\displaystyle\sum}
\newcommand{\Prod}{\displaystyle\prod}
\newcommand{\Int}{\displaystyle\int}
\newcommand{\dint}{\, \mathrm{d}}
\newcommand{\Lim}{\displaystyle \lim}
\newcommand{\Frac}{\displaystyle\frac}

\newcommand{\Nc}{N_{cont}}
\newcommand{\Ni}{N_{int}}
\newcommand{\Ne}{N_{estrela}}

\newcommand{\Dparc}[2]{\dfrac{\partial #1}{\partial #2}}
\newcommand{\Dparcn}[3]{\dfrac{\partial^#3 #1}{\partial^#3 #2}}

\newcommand{\R}{\mathbb{R}}
\newcommand{\V}{\mathcal{V}}
\newcommand{\I}{I_u}
\newcommand{\Fi}{\varphi}
\newcommand{\se}{\mbox{ se }}
\newcommand{\norma}[1]{\left|\left| #1 \right|\right|}
\newcommand{\sistema}[1]{ \left\{ #1 \right. }

\newtheorem{exemplo}{\pr \sc Exemplo}[section]%[chapter]
\newtheorem{defi}{\pr \sc Defini\c{c}\~ao}[section]%[chapter]
\newtheorem{obs}{\pr \sc Observa\cao}[section]%[chapter]
\newtheorem{teor}{\pr \sc Teorema}[section]%[chapter]
\newtheorem{lema}{\pr \sc Lema}[section]%[chapter]
\newtheorem{prop}{\pr \sc Proposi\cao}[section]%[chapter]
\newtheorem{exercise}{\pr \sc Exerc\'\i cios}[section]%[chapter]
\newtheorem{alg}{\pr Algoritmo}[section]%[chapter]

\setlength{\columnsep}{1cm}

\setlength{\columnsep}{1cm}

%basicstyle=\footnotesize\ttfamily,


\begin{document}

\thispagestyle{empty}
\begin{center}
	UNIVERSIDADE DE SÃO PAULO – USP
	
	INSTITUTO DE CIÊNCIAS MATEMÁTICAS E DE COMPUTAÇÃO
	
	DEPARTAMENTO DE SISTEMAS DE COMPUTAÇÃO
	
	\vspace{7cm}
	
	\Large{\textbf{RELATÓRIO}}
	 
	\Large{\textbf{Projeto 3 - Fibonacci em $O\log_n$}}
	
	\vspace{6cm}
	
	Alunos: Adams Vietro Codignotto da Silva - $6791943$ \\ 
			Guilherme da Silva Biondo - $8124267$
	
	\vspace{6cm}
	
	São Carlos
	
	2013
\end{center}

\newpage

\pagenumbering{arabic}

\section{Introdução}

O trabalho foi implementado utilizando as técnicas fornecidas na descrição do mesmo. Foi desenvolvido utilizando a linguagem C, no Linux Xubuntu 13.10 64bits. O arquivo contem um arquivo \emph{Makefile} para compilação do programa, usando os comandos \emph{make all2} para compilação inicial, \textit{make all} para compilações futuras e \emph{make run} para execução do mesmo.

\section{Modelagem do problema}	
\subsection{Matriz base de Fibonacci}
Primeiramente foi necessário encontrar a matriz base para o cálculo da Sequência de Fibonacci. Vamos mostrar que a matriz 
$\bigl(\begin{smallmatrix}
1&1\\ 1&0
\end{smallmatrix} \bigr)$
é a matriz utilizada para o cálculo dos termos de Fibonacci. 
Temos inicialmente a sequência definida por $F_n=F_{n-1} + F_{n-2}$. Da combinação linear, podemos calcular $F_2$ como:
\begin{equation}\nonumber
     \begin{bmatrix}
       2\\ 1\\
     \end{bmatrix}
     =
     \begin{bmatrix}
       F_2\\ F_1\\
     \end{bmatrix}
     =
     \begin{bmatrix}
     1&1 \\ 1&0 \\
     \end{bmatrix}
     \begin{bmatrix}
     F_1\\ F_0\\
     \end{bmatrix}
     =
     \begin{bmatrix}
     1&1 \\ 1&0 \\
     \end{bmatrix}
     \begin{bmatrix}
       1\\ 1\\
     \end{bmatrix}
     =
     A      \begin{bmatrix}
       1\\ 1\\
     \end{bmatrix}
\end{equation}
Podemos calcular $F_3$ conforme a equação a seguir, denotando A = $\bigl(\begin{smallmatrix}
1&1\\ 1&0
\end{smallmatrix} \bigr)$:
\begin{equation}\nonumber
\begin{bmatrix}
3\\ 2\\
\end{bmatrix}=
\begin{bmatrix}
F_3\\ F_2\\
\end{bmatrix}=
A \times \begin{bmatrix}
F_2\\ F_1\\
\end{bmatrix}=
A \times A \times \begin{bmatrix}
F_1\\ F_0\\
\end{bmatrix}= A^2 \begin{bmatrix}
1\\ 1\\
\end{bmatrix}
\end{equation}
Então generalizando, temos:
\begin{equation}\nonumber
\begin{bmatrix}
F_{n+1}\\
F_{n}\\
\end{bmatrix}
=
A^n \times \begin{bmatrix}
       1\\ 1\\
\end{bmatrix}
\end{equation}
\subsection{Algoritmo de Strassen}
Também utilizamos o Algoritmo de Strassen para efetuar a multiplicação de matrizes, uma vez que este também faz parte de métodos de Divisão e Conquista. Este algoritmo se resume em realizar a multiplicação de matrizes 2x2 em $O(n^2)$ ao invéz do usual $O(n^3)$.
Dadas duas matrizes A=$\bigl(\begin{smallmatrix}
a&b\\ c&d
\end{smallmatrix} \bigr)$ e B=$\bigl(\begin{smallmatrix}
e&f\\ g&h
\end{smallmatrix} \bigr)$, temos que A$\times$B pode ser resumido em:
\begin{equation} \nonumber
\begin{array}{lcl}
P_1 = a\times(f-h) & & r = P_5 + P_4 - P_2 + P_6\\
P_2 = (a+b)\times h & & s = P_1 + P_2\\
P_3 = (c+d)\times e & & t = P_3 + P_4\\
P_4 = d \times (g-e) & & u = P_5 + P_1 - P_3 - P_7\\
P_5 = (a+d) \times (e+h) & & \\
P_6 = (b-d) \times (g+h) & & \\
P_7 = (a-c) \times (e+f) & & \\
\end{array}
\end{equation}

\begin{equation}\nonumber
A \times B = 
\begin{bmatrix}
r&s\\t&u
\end{bmatrix}
\end{equation}
Este algoritmo utiliza 7 multiplicações e 18 adições/subtrações, resultando em $T(n) = 7T(n/2)+\Theta(n^2)$, uma vez que este divide as matrizes A e B em duas submatrizes $(n/2) \times (n/2)$.
\subsection{Potenciação}
Normalmente se calcula $a^n$ em $O(n)$. Utilizando o algoritmo de divisão e conquista provido nas instruções do trabalho, temos que $T(n)=T(n/2) + \Theta (1) \Rightarrow T(n) = \Theta (\log_n)$.
Note que, para facilitar a implementação, utilizamos a propriedade
\begin{equation}
(x^{n/2})^2 = x^{n/2}*x^{n/2} = (x^2)^{n/2}
\end{equation}

\section{Experimentos e Resultados}
	
Durante a implementação do código, foi necessário o uso de \emph{long long int}, uma vez que a entrada é de tamanho $10^{11}$. Como utilizamos varias matrizes 2x2, o uso de memória tornou-se gigantesco, então todas as estruturas foram declaradas estaticamente. Deste modo evitamos o uso excessivo de memória ao alocarmos dinâmicamente as matrizes, pois com o uso de alocação dinâmica várias matrizes são criadas e usadas como parâmetro de retorno, assim não podendo ser desalocadas e gerando o uso excessivo de memória.
A execução foi extremamente rápida, levando menos de 1 segundo para qualquer uma das entradas no intervalo definido.

\section{Conclusões}
A combinação de todos os algoritmos (Strassen ($O(n^2)$) e Potenciação em $O(Log_2n)$) resultou em um programa de $\simeq O(Log_n)$. Fez-se a saída ser o resto da divisão de $F_n$ por $10^6$ ($F_n  \mbox{mod} 10^6$) pelo fato de que ao realizarmos as multiplicações, iria ocorrer o overflow de variáveis nas matrizes. Então a cada iteração, realiza-se o mod $10^6$ sobre cada multiplicação durante a execução do algoritmo de Strassen, uma vez que todos os casos sempre iniciam-se com
$\bigl(\begin{smallmatrix}
1&1\\ 1&0
\end{smallmatrix} \bigr)$.

O método utilizado no trabalho mostrou-se muito mais eficaz que sua versão iterativa ou recursiva, que possuem $O(n)$. Entretanto, este método só é mais eficiente pela forma com que calculamos a potênciação das matrizes, uma vez que se não utilizassemos a Potenciação em $O(Log_2n)$ este método degeneraria para sua versão iterativa/recursiva (e ainda pior, pois utilizamos muito mais memória para os cálculos).
\end{document}
